\chapter{Conclusion}
\section{Completion of Initial Objectives}
The goal of the project was to develop, evaluate and deploy an Android app with the purpose of helping users with early detection of skin cancer. The project's focus revolved around the successful completion of the main phases in mobile app development.

As per the technical tasks, the app allows adding new spots or new photos of a spot. The app also allows comparing two images of a spot side by side. These images can then be emailed to a user specified email address from within the app. Information screens are also presented to users on the app's home screen. To accommodate these core features, the design and implementation stages were carried out, delving into the process and decisions behind the app's UI and implementation.

Secondly, the app's evaluation was completed through user testing and usability evaluations. This process was done through stages of internal, Alpha, and Beta testing. The app's usability was evaluated through the use of think-aloud studies, interviews, system usability scale assessments and preference tests.

Thirdly, the publishing tasks of the app included deploying a website to guide users in performing all of the core features of the app. Lastly, a stable version of the app was published on the Google Play Store, acting as a milestone for the successful completion of the project.


\section{Future Work}
With the unique nature of app development, it can be sometimes difficult to deem a project as complete. There are always new ideas to be implemented, and the rapid growth of the Android platform will guarantee to bring new bugs and incompatibilities to every app. Leaving this aside, if the project was to be continued, the following areas would be sensible to explore:
\begin{itemize}
    \item \textbf{Reminder Notifications} - Implementing reminders for the user to take a new photo of a spot. These could range from push notifications to emails or SMS messages.
    \item \textbf{Accomodate for Mistakes} - Among others, this could include: allowing users to modify a photo's crop, deleting, moving, or renaming spots.
    \item \textbf{In-App Help} - Providing information on app actions and buttons on each screen of the app. This could be done with libraries such as \emph{TourGuide} \cite{rong_2018}.
    \item \textbf{Image Matching Function} - Exploring the image processing side of the project. Developing a feature that takes two images of a spot and makes them more similar, this facilitates the comparison process. A starting point would be to automatically adjust rotation, scale and colors of both images.
\end{itemize}



